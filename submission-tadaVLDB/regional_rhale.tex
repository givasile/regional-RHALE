%% The first command in your LaTeX source must be the \documentclass command.
%%
%% Options:
%% twocolumn : Two column layout.
%% hf: enable header and footer.
\documentclass[
twocolumn,
% hf,
]{ceurart}

%%
%% One can fix some overfulls
\sloppy

%%
%% Minted listings support 
%% Need pygment <http://pygments.org/> <http://pypi.python.org/pypi/Pygments>
\usepackage{listings}
%% auto break lines
\lstset{breaklines=true}

\usepackage{blindtext}
\usepackage[linesnumbered,ruled]{algorithm2e}
\usepackage{amsmath}
\usepackage{tikzsymbols}
\usepackage{pifont} % for \ding command
\usepackage{listings}
\usepackage{multirow}
\usepackage{bm}
\usepackage{bbm}
\usepackage{xfrac}
\usepackage{booktabs} % For better horizontal rules
\usepackage{cellspace} % For adjusting vertical spacing
\usepackage{xcolor}
\newcommand\todop[1]{\textcolor{red}{#1}} % todo in the paper
\newcommand\todos[1]{\textcolor{blue}{#1}} % todo in the software
\newcommand{\tik}{\textcolor{green}{\ding{51}}}
\newcommand{\ntik}{\textcolor{red}{\ding{55}}}
\usepackage{lastpage}
\usepackage{caption}
\usepackage{subcaption}

\newcommand{\xcc}{\mathbf{x_{/s}}}
\newcommand{\Xcal}{\mathcal{X}}
\newcommand{\xb}{\mathbf{x}}
\newcommand{\xc}{\mathbf{x_c}}
\newcommand{\Xc}{\mathbf{X_c}}
\newcommand{\xci}{\mathbf{x}^i_c}



%%
%% end of the preamble, start of the body of the document source.
\begin{document}

%%
%% Rights management information.
%% CC-BY is default license.
\copyrightyear{2024}
\copyrightclause{Use permitted under Creative Commons License Attribution 4.0 International (CC BY 4.0).}

%%
%% This command is for the conference information
\conference{The 2nd World Conference on eXplainable Artificial Intelligence,
  July 17--19, 2024, Malta, Valetta}

%%
%% The "title" command
\title{Effector: A Python package for regional explanations}

%%
%% The "author" command and its associated commands are used to define
%% the authors and their affiliations.
\author[1,2]{Vasilis Gkolemis}
\address[1]{Harokopio University of Athens}
\address[2]{ATHENA Research Center}
\author[1]{Christos Diou}
\author[3]{Eirini Ntoutsi}
\address[3]{University of the Bundeswehr Munich}
\author[2]{Theodore Dalamagas}
\author[4]{Bernd Bischl}
\address[4]{Munich Center for Machine Learning (MCML), Department of Statistics, LMU Munich}
\author[4]{Julia Herbinger}
\author[4]{Giuseppe Casalicchio}


%%
%% The abstract is a short summary of the work to be presented in the
%% article.
\begin{abstract}
Global feature effect methods explain a model outputting one plot per feature. The plot shows the average effect of the feature on the output, like the effect of age on the annual income.  However, average effects may be misleading when derived from local effects that are heterogeneous, i.e., they significantly deviate from the average.  To decrease the heterogeneity, regional effects provide multiple plots per feature, each representing the average effect within a specific subspace.  For interpretability, subspaces are defined as hyperrectangles defined by a chain of logical rules, like age's effect on annual income separately for males and females and different levels of professional experience.  We introduce \texttt{Effector}, a Python library dedicated to regional feature effects.  \texttt{Effector} implements well-established global effect methods, assesses the heterogeneity of each method and, based on that, provides regional effects. \texttt{Effector} automatically detects subspaces where regional effects have reduced heterogeneity.  All global and regional effect methods share a common API, facilitating comparisons between them. Moreover, the library's interface is extensible so new methods can be easily added and benchmarked. The library has been thoroughly tested, ships with many tutorials (\href{https://xai-effector.github.io/}{https://xai-effector.github.io/}) and is available under an open-source license at PyPi \href{https://pypi.org/project/effector/}{https://pypi.org/project/effector/} and Github \href{https://github.com/givasile/effector}{https://github.com/givasile/effector}.
\end{abstract}

%%
%% Keywords. The author(s) should pick words that accurately describe
%% the work being presented. Separate the keywords with commas.
\begin{keywords}
  Explainability \sep
  Interpretability \sep
  Feature Effect \sep
  Regional Effect \sep
  Global Explanations
\end{keywords}

%%
%% This command processes the author and affiliation and title
%% information and builds the first part of the formatted document.
\maketitle

\section{Introduction}
\label{sec:introduction}

The increasing adoption of machine learning (ML) in high-stakes domains like healthcare and finance has raised the demand
for explainable AI (XAI) techniques~\citep{freiesleben_scientific_2022, ribeiro2016should}.
Global feature effect methods explain a black-box model through a set of plots, where each plot is the effect of a feature on the output, as in Figure~\ref{subfig:a}.

Global effects may be misleading when the black-box model $f(\cdot)$ exhibits interactions between features.
An interaction between two features, $x_s$ and $x_k$,
%EIRINI: xs, xk should be introduced properly Vasilis: Done!
exists when the difference in the output $f(\mathbf{x})$ as a result of changing the value of $x_s$ \textit{depends} on the value of $x_k$~\citep{friedman_predictive_2008}.
Global effects are often computed as averages over local effects.
% For example, individual conditional expectation (ICE) curves~\citep{goldstein_peeking_2014} are the local counterparts of partial dependence plots (PDP) see also example in Fig.~\ref{fig:main-concept}.
When feature interactions are present, local effects become heterogeneous, i.e., they significantly deviate from the average (global) effect.
In these cases, the global effect may be misleading, a phenomenon known as \emph{aggregation bias}~\citep{mehrabi_survey_2021, herbinger_repid_2022}.

Regional~\citep{herbinger2023decomposing, herbinger_repid_2022, molnar2023model, britton2019vine, hu2020surrogate, scholbeck2022marginal} or cohort explanations~\citep{sokol2020explainability}, partition the input space into subspaces and compute a regional explanation within each.
The partitioning aims at subspaces with homogeneous local effects, i.e., with reduced feature interactions, yielding regional effects with minimized aggregation bias~\citep{herbinger_repid_2022}. 
By combining these regional explanations, users can interpret the model's behavior across the entire input space.
Several libraries focus on XAI, but
none targets on regional effect methods (Table~\ref{tab:package-comparison}). Therefore, we present \texttt{Effector}, a Python library dedicated to regional explainability methods, which:

\begin{itemize}
\item implements well-established global and regional effect methods, accompanied by an intuitive way to visualize the heterogeneity of each plot. 
\item follows a consistent and modular software design. Existing methods share a common API and novel methods can be easily added and compared to existing ones.
\item demonstrates through tutorials the use of regional effects in real and synthetic datasets.
\end{itemize}



\section{Conclusion and Future Work}

\bibliography{regional_rhale.bib}

\end{document}

%%
%% End of file
